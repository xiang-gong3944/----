\documentclass[twocolumn,10pt,dvipdfmx,a4paper]{jsarticle}

\usepackage[margin=15mm]{geometry} %余白の削除

\usepackage{amsmath,amssymb}
\usepackage{bm}
\usepackage[dvipdfmx]{graphicx}
\usepackage{physics} % http://mirrors.ibiblio.org/CTAN/macros/latex/contrib/physics/physics.pdf
\usepackage{siunitx} %SI単位を楽に出力
\usepackage{mathtools} %環境の追加
% \usepackage{circuitikz} %電気回路をtex中で書く
% \usepackage{caption} %番号なしキャプションを書く
% \usepackage{cancel} %式中に斜線を入れる
% \usepackage{tensor} %テンソルの添え字を書く
% \usepackage{tikz} %図を書く
% \usepackage{ascmac} %四角い枠の中に文章を書く
% \usepackage{float} %figureで[hbp]オプションを使う
% \usepackage{hyperref}  \usepackage{pxjahyper} %ハイパーリンクをつかう
% \usepackage{tablefootnote} %表中に注釈をいれる
% \usepackage[thicklines]{cancel} %数式中の取り消し線
% \usepackage[version=4]{mhchem} %化学式の入力
\usepackage{pdfpages}
% \usepackage{wrapfig} %文章の回り込み
\usepackage[subrefformat=parens]{subcaption} %(a)図のようにすることができるやつ
\usepackage{here}
\usepackage{url}
\usepackage{mathrsfs}

\graphicspath{{./image/}}

\renewcommand{\abstractname}{Abstract}

\title{量子多体系とニューラルネットワーク}
\author{1522068 西原翔}
\date{\today}

\begin{document}
\twocolumn[
\maketitle

\begin{abstract}
    The application of neural networks in condensed matter physics
     is transforming the field by handling complex and large datasets.
     Neural networks facilitate rapid material discovery
     by predicting physical properties like thermal and electrical conductivity from vast datasets,
     significantly speeding up the process compared to traditional methods.

     In this report, I'm going to introduce the representation of wave functions
     using neural nettworks.
\end{abstract}]
\section{始めに}
SGC ライブラリ「量子多体系とニューラルネットワーク」\cite{SGC191}
を主に参考にしている。

\bibliographystyle{junsrt}
\bibliography{reference}

\end{document}