\documentclass[twocolumn,10pt,dvipdfmx,a4paper]{jsarticle}

\usepackage[margin=15mm]{geometry} %余白の削除

\usepackage{amsmath,amssymb}
\usepackage{bm}
\usepackage[dvipdfmx]{graphicx}
\usepackage{physics} % http://mirrors.ibiblio.org/CTAN/macros/latex/contrib/physics/physics.pdf
\usepackage{siunitx} %SI単位を楽に出力
\usepackage{mathtools} %環境の追加
% \usepackage{circuitikz} %電気回路をtex中で書く
% \usepackage{caption} %番号なしキャプションを書く
% \usepackage{cancel} %式中に斜線を入れる
% \usepackage{tensor} %テンソルの添え字を書く
% \usepackage{tikz} %図を書く
% \usepackage{ascmac} %四角い枠の中に文章を書く
% \usepackage{float} %figureで[hbp]オプションを使う
% \usepackage{hyperref}  \usepackage{pxjahyper} %ハイパーリンクをつかう
% \usepackage{tablefootnote} %表中に注釈をいれる
% \usepackage[thicklines]{cancel} %数式中の取り消し線
% \usepackage[version=4]{mhchem} %化学式の入力
\usepackage{pdfpages}
% \usepackage{wrapfig} %文章の回り込み
\usepackage[subrefformat=parens]{subcaption} %(a)図のようにすることができるやつ
\usepackage{here}
\usepackage{url}
\usepackage{mathrsfs}

\graphicspath{{./image/}}

\renewcommand{\abstractname}{Abstract}

\title{量子多体系とニューラルネットワーク}
\author{1522068 西原翔}
\date{\today}

\begin{document}
\twocolumn[
\maketitle

\begin{abstract}
    The application of neural networks in condensed matter physics
     is transforming the field by handling complex and large datasets.
     Neural networks facilitate rapid material discovery
     by predicting physical properties like thermal and electrical conductivity from vast datasets,
     significantly speeding up the process compared to traditional methods.

     In this report, I'm going to introduce the representation of wave functions
     using neural networks.
\end{abstract}]
\section{始めに}
SGC ライブラリ「量子多体系とニューラルネットワーク」\cite{SGC191}
のなかで、ボルツマンマシンを用いた量子多体系の紹介があった。
ただ、このテキストでは大まかな紹介になっていて、
具体的な話は元の Carleoらの元の論文など\cite{Carleo-2017}\cite{Carleo-2018}を見るようにというようにあった。
そこで、このレポートはCarleo の元の論文\cite{Carleo-2017}を追っていくつもりである。

\section{量子多体系の扱われ方}
実際にニューラルネットワークがどのように量子多体系に応用されているかを見る前に、
そのモチベーションを見ていこう。

量子力学において系の状態はハミルトニアンと呼ばれる演算子の固有値方程式で表される。
固有ベクトルは波動関数と呼ばれ、固有値は系のエネルギーとなる。
量子一体問題の例として、
電子がクーロン引力によって陽子に束縛される水素原子原子を考える。
\footnote{実はこれは二体問題であるのだが、
重心運動と相対運動に分離することができ、相対運動にだけ注目すると一体問題になる。}
この系の状態は次のシュレディンガー方程式と呼ばれる固有値方程式で書かれる。
\begin{equation}
    \qty(-\frac{1}{2}\laplacian -\frac{1}{r})\psi(\vb*{r})=E\psi(\vb*{r})
\end{equation}
\footnote{ここでは原子単位系と呼ばれる単位系を用いて無次元化している。}
この時点で複雑な偏微分方程式になっていることがわかる。

では\(N_e\)個の電子が結晶中に固定された\(N_n\)個原子核からのクーロンポテンシャルを受けながら、
電子同士もクーロン相互作用をするような状態の固有値方程式は次のようになる。
\begin{align}
    \sum_{i=1}^{N_n}\sum_{j=1}^{N_e}\qty(-\frac{1}{2}\laplacian_j
    -\frac{1}{\abs{\vb*{R}_i-\vb*{r}_j}}
    +\sum_{k=1}^{N_e}\frac{1}{\abs{\vb*{r}_j-\vb*{r}_k}})\psi = E\psi
    \label{eq:no2}
\end{align}
というようにとても手計算で扱えたものにはなっていない。
また、計算機を使ったものであっても粒子数のべきのオーダーで計算量が増えていく。
そのため、身近にあるパソコンでは 50 個程度電子の系でしかこの計算をすることができない。
これでは実際の物質中のような電子が\num{6.0e23}個もあるような系は到底理理解できないように見える。

しかし物性物理の理論では物理的な解釈をもとにこの偏微分方程式に近似を加えることでさまざまな現象を説明する。
その近似の例を見てみよう。
電子同士の相互作用を無視して、原子核からのクーロン引力ポテンシャルを周期的でとても弱いものとして扱う。
そして、電子の質量に周期的なポテンシャルを繰り込むことで最終的には
\begin{equation}
    -\frac{1}{2m^*}\laplacian F(\vb*{r}) = E F(r)
\end{equation}
という微分方程式にすることができる。
このような近似をしていくのがバンド理論と呼ばれるものである。
これにより金属のとても多くの性質を説明することができる。

これは驚くべきことである。
つまり元の微分方程式ではこまごまとした余計な情報が多く含まれていて、
系の状態を決定づける重要な特徴量は微分方程式を完全に解かずとも得られることを意味する。
従来の物理学ではこの特徴量を理論体系とそれに基づく勘によって探し出していた。
この多自由における特徴量を求めるといった問題はまさに近年発展している機械学習の得意とする問題である。
なので物性物理に限らず物理学の世界において理論物理、実験物理に加え、
新しく計算物理と呼ばれる新領域も物理学会にて作られるようになった。

\section{ニューラルネットワーク量子状態}
スピンやボゾン粒子の数といった\(N\)個の離散的な自由度\(\mathcal{S}=(S_1,\,S_2,\,\dots\, S_N)\)をもった量子系を考える。
関数は定義域から値域への写像だということを思い出すと、
この系の波動関数を\(N\)次元の集合\(\mathcal{S}\)というのをある複素数へと移すものだと考えることはできる。
実際、波動関数の形を求めることで何かを得るということは少なくあるエネルギーを持った粒子がどれほどあるかといった情報しか使わない。
なので関数の形はわからずともこと足りる。
この観点から波動関数は完全なブラックボックスとして扱っても問題ないことがわかる。
入力をブラックボックスに通して出力を得るというのは人工ニューラルネットワークモデルのやることである。
以上より波動関数をニューラルネットワークを用いて表すというアイデアが生まれる。

ニューラルネットワークのモデルというのはさまざまある。
一口に量子多体系といってもそれぞれのモデルがどの問題に適しているかというのはもちろん違う。
今回は横磁場イジング模型と反強磁性ハイゼンベルグ模型というのを扱う。
この問題を扱う際には制限ボルツマンマシンというニューラルネットワークモデルを用いる。

\section{ボルツマンマシン}


\bibliographystyle{junsrt}
\bibliography{reference}

\end{document}