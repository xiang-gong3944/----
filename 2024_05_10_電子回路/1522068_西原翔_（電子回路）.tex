\documentclass[11pt,dvipdfmx,a4paper]{jsarticle}

\usepackage{amsmath,amssymb}
\usepackage{bm}
\usepackage[dvipdfmx]{graphicx}
\usepackage{physics} % http://mirrors.ibiblio.org/CTAN/macros/latex/contrib/physics/physics.pdf
\usepackage{siunitx} %SI単位を楽に出力
\usepackage{mathtools} %環境の追加
\usepackage{circuitikz} %電気回路をtex中で書く
% \usepackage{caption} %番号なしキャプションを書く
% \usepackage{cancel} %式中に斜線を入れる
% \usepackage{tensor} %テンソルの添え字を書く
% \usepackage{tikz} %図を書く
% \usepackage{ascmac} %四角い枠の中に文章を書く
% \usepackage{float} %figureで[hbp]オプションを使う
% \usepackage{hyperref}  \usepackage{pxjahyper} %ハイパーリンクをつかう
% \usepackage{tablefootnote} %表中に注釈をいれる
% \usepackage[thicklines]{cancel} %数式中の取り消し線
% \usepackage[version=4]{mhchem} %化学式の入力
\usepackage{pdfpages}
% \usepackage{wrapfig} %文章の回り込み
\usepackage[subrefformat=parens]{subcaption} %(a)図のようにすることができるやつ
\usepackage{here}
\usepackage[margin=15mm]{geometry} %余白の削除

\graphicspath{{./image/}}

\begin{document}

%出力したpdfを表紙にするとき
% \includepdf[pages=1,noautoscale=false]{cover.pdf}
% \newpage

%texで表紙を書くとき
\quad\\[35mm]
\centerline{\Huge{\textsf{第 2 回}}}
\quad\\[5mm]
\centerline{\Huge{\textsf{応 用 物 理 学 実 験}}}
\quad\\[5mm]
\begin{table}[h]
	\centering
	\begin{tabular}{| c | c |}
		\hline
		\Huge\textsf{{題目}} & \Huge{\textsf{電子回路}} \rule[-5mm]{0mm}{15mm} \\
		\hline
	\end{tabular}
\end{table}
\quad\\[10mm]
\begin{table}[h]
	\centering
	\begin{tabular}{l l}
		\hline
		\LARGE{\textsf{氏\qquad 名}} & \LARGE{\textsf{: 西原 翔}} \rule[0mm]{0mm}{6mm} \\
		\hline
		\LARGE{\textsf{学  籍  番  号}} & \LARGE{\textsf{: 1522068}} \rule[0mm]{0mm}{6mm} \\
		\LARGE{\textsf{学部学科学年}} & \LARGE{\textsf{: 理学部第一部応用物理学科3年}}\\
		\hline
	\end{tabular}
\end{table}
\quad\\[10mm]
\centerline{\LARGE{\textsf{共同実験者:1522064 中井空弥}}}\\[2mm]
\quad\\[10mm]
\centerline{\LARGE{\textsf{提出年月日:2024年05月30日}}}\\[2mm]
\centerline{\LARGE{\textsf{実験実施日:2024年05月10日}}}\\[2mm]
\centerline{\LARGE{\textsf{\qquad\qquad\quad\;2024年05月17日}}}
\quad\\[10mm]
\centerline{\LARGE{\textsf{東 京 理 科 大 学 理 学 部 第 1 部}}}\\[2mm]
\centerline{\LARGE{\textsf{応 用 物 理 学 教 室}}}

\thispagestyle{empty}
\clearpage
\addtocounter{page}{-1}
\newpage

% \twocolumn
\section{目的}
オペアンプは電子回路で用いられる素子の一つで、
二つの入力電圧を受け取りその差を大きな倍率で出力する。
この性質により、入力信号の増幅とフィルタリングであったり、加算、減算、微分といった計算、
出力をまた入力に戻してフィードバックループを作って制御に用いられるなど重要な役割を果たしている。

オペアンプには回路の解析が簡単になるよい性質があり、その性質を用いてこの素子を用いた回路設計を行っていくことが多い。
しかしこれは一種の近似であるため、この手法が成り立たない領域がある。

この実験ではオペアンプを活用した回路である反転増幅回路、ボルテージ・フォロワ回路、2 次のローパスフィルタ回路の 3 つ回路の振舞いを実際に測定した。

\section{原理}
\subsection{オペアンプ}
% オペアンプの歴史を電子回路の本から引っ張ってくる

オペアンプには五つの端子があり模式図と、
今回の実験で実際に使う素子 LM741 は図\ref{fig:no1} のようになっている。
2 番ピンの反転入力端子の電圧を \(V_{-}\), 3 番ピンの非反転入力端子の電圧を\(V_{+}\), 6 番ピンの出力端子の電圧を \(V_o\) とする。
電圧増幅度を \(\mu\simeq 10^5\) としたとき、これらの間には
\begin{equation}
	V_o = \mu (V_{+} - V_{-})
\end{equation}
となっている。
電圧を増幅するためのエネルギー減として
出力電圧は 4 番ピンの負電源と 7 番ピンの正電源をつなぐ必要がある。
これらの端子を超える電圧を出力することはできない。(課題1)
このような状態を出力飽和という。

出力端子を入力端子につなげることを考える。
このとき入力電圧と出力電圧はほぼ同じオーダーとなるため、
大きな電圧増幅度との積をとる\(V_{+} - V_{-}\)この値は 電圧増幅度の逆数のオーダーでなければならない。
\(\mu\simeq10^5\) であるため、\(V_{+}=V_{-}\) と言ってもよい。
つまり反転入力端子と非反転入力端子が短絡されているとみなせる状態となっている。
このように反転入力端子と非反転入力端子の電圧を考えるのを仮想短絡と呼ぶ。

また、理想的なオペアンプは入力に関しては電圧だけを見て入力部の回路には干渉せず、
出力は決まった電圧を出すため電流を吸い込むものと考える。
これは入力については電圧計と同様に入力インピーダンスを限りなく大きく、
出力は電圧源と同様に出力インピーダンスは限りなく小さいものとみなす。

この二つの性質はある一定の条件における近似になっているので成り立つときと成り立たないときに気を付けて
回路の設計をしなければならない。

\subsection{反転増幅回路}
入力電圧を逆位相で増幅し出力する回路を反転増幅回路と言い、
オペアンプを使ったものとしては図\ref{fig:no2}がある。
仮想短絡により
\begin{equation}
	V_{-} = 0 \text{ V}
\end{equation}
である。
この回路に入力された電流はオペアンプの入力端子に流れず、\(R_{f}\) にすべて流れる。
そのため\(R_i\) の抵抗に流れる電流と、\(R_f\) の抵抗に流れる電流について等式を立てて整理していくと、
\begin{align}
	\frac{V_i-V_{-}}{R_i} &= \frac{V_{-}-V_o}{R_f}\\
	V_0 &= -\frac{R_f}{R_i} V_i
\end{align}
というようになる。

\subsection{ボルテージ・フォロワ回路回路}
回路を設計する際、入力となる部分の回路と出力となる部分の回路を干渉させたくないことがある。
そういったときに入力には電圧だけを読み取り電流を吸い込まず、
出力では決まった電圧を出力し必要に応じて電流を出すといった回路が求められる。
そういった回路をバッファ回路という。
理想オペアンプは入力は電圧だけ読み取り電流は流さず出力は決まった電圧を出力する素子であるため、
バッファ回路として使われることがある。
図\ref{fig:no3}のように反転入力端子と出力を結びつけると、
非反転入力端子には電流が流れ込まず、
仮想短絡により入力が等倍で出力されていることがわかる。

実際のオペアンプは IC 部分の応答の速さにより入力信号が瞬時に出力されない。
この実験では 2 種類のオペアンプ LM741 と TL071 の応答の様子を確認した。


\subsection{2次のローパス・フィルタ回路}
交流入力があったとき、特定の周波数だけを取り除く回路をフィルタ回路といい、
とくに高周波成分を取り除く回路をローパスフィルタ退路という。
その例が図\ref{fig:no4} である。
これについて解析していく。(課題2)
仮想短絡により
\begin{equation}
	V_{+} = V_{-} = V_o. \label{eq:low-pass-1}
\end{equation}
電圧\(V_1\)と書かれた地点において電流の流出入を考えると
\begin{equation}
	\frac{V_i-V_1}{R_1}+\frac{V_{-}-V_1}{1/sC_1} = \frac{V_1-V_{+}}{R_2} \label{eq:low-pass-2}.
\end{equation}
また電圧\(V_{+}\)と書かれた地点において電流の流出入を考えるとオペアンプには電流が流れ込まないので
\begin{equation}
	\frac{V_1-V_{+}}{R_2} = \frac{V_{+}}{1/sC_2} \label{eq:low-pass-3}
\end{equation}
となる。これらの式を整理してを入力と出力の比である伝達関数\(G(s)\)を求める。
(\ref{eq:low-pass-1}) 式と(\ref{eq:low-pass-3}) 式より
\begin{equation}
	V_1 = V_o \qty(1+sC_2 R_2).
\end{equation}
これを (\ref{eq:low-pass-3})式に入れていくと
\begin{align}
	\frac{V_i-V_1}{R_1}+\frac{V_{o}-V_1}{1/sC_1} &= \frac{V_{o}}{1/sC_2}\\
	V_i &= V_o \biggl\{(C_1C_2R_1R_2)s^2+(R_1+R_2)C_2s+1\biggr\}\\
	G(s) &= \frac{V_o}{V_i} = \frac{\omega_c^2}{s^2+2\zeta\omega_c s+\omega_c^2}
\end{align}
となる。ここで共振周波数\(\omega_c\) と制動係数\(\zeta\) は次のとおりである。
\begin{align}
	\omega_c &:= \frac{1}{\sqrt{C_1C_2R_1R_2}},\\
	\zeta &:= \frac{1}{2}\sqrt{\frac{C_2}{C_1}}\qty(\sqrt{\frac{R_1}{R_2}}+\sqrt{\frac{R_2}{R_1}}).
\end{align}


\section{実験}
\subsection{反転増幅回路}
図\ref{fig:no2} の反転増幅回路をブレッドボード上に組んでその振る舞いを調べた。
オペアンプは LM741 を使い、オペアンプに \(\pm 15.0 V\) の電圧を供給した。
また抵抗 \(R_f\) は 100 k\si{\ohm} として、\(R_i\) は各測定で切り替えた。
\subsubsection{直流特性}
直流特性を調べるため入力として -10 V から 10 V の電圧をかけたときの
反転入力端子の電圧\(V_{-}\) と出力電圧\(V_{o}\) を
\(R_i =\) 33 k\si{\ohm} のときと
\(R_i =\) 68 k\si{\ohm} のときの両方を測定した。

\subsubsection{交流電圧特性}
交流電流の振幅の入出力特性を調べた。
\(R_i =\) 33 k\si{\ohm} のときと
\(R_i =\) 68 k\si{\ohm} のときそれぞれにおいて、
10 kHz の交流電圧を入力したときに実効値を 0 V から 10 V まで変えたときの出力電圧を測定した。

\subsubsection{交流周波数特性}
\(R_i =\) 33 k\si{\ohm} のときと
\(R_i =\) 15 k\si{\ohm} のときそれぞれにおいて、
実効値 1.0 V の交流電圧の周波数を 100 Hz から 100 kHz まで変化させたときの出力電圧を測定した。

\subsection{ボルテージ・フォロワ回路}
図\ref{fig:no3}のボルテージ・フォロワ回路をブレッドボード上に組み、
10 kHz で peak-to-peak が 10V の矩形波を入力したときのオペアンプの種類による応答の波形の違いを測定した。
このとき用 LM741 と TL071 のオペアンプを使用し\(\pm 15.0 V\) の電圧を供給した。

\subsection{2 次のローパス・フィルタ回路}
図\ref{fig:no4}の 2 次のローパス・フィルタ回路をブレッドボード上に組み、
伝達関数の制動係数\(\zeta\)を変えたときにていったときのゲインと位相の周波数依存性を 10 Hz から 20 kHz まで測定した。
また 300 Hz で peak-to-peak が 1V の矩形波を入力したときの波形
このとき位相はオシロスコープについている機能を用いて測定した。
使用したオペアンプは LM741 で、\(\pm 15.0 V\) の電圧を供給した。
制動係数を変えるのに使った抵抗とコンデンサのインピーダンスは表\ref{table:no1}のようになっている。
\begin{table}
	\centering
	\caption{2次のローパス・フィルタ回路(図\ref{fig:no4})に使用した抵抗とコンデンサのパラメータ}
	\label{table:no1}
	\begin{tabular}[t]{cccccc}
		\hline
		\(R_1\) (k\si{\ohm}) & \(R_2\) (k\si{\ohm}) & \(C_1\) (nF) & \(C_2\) (nF) & \(\omega_c\) (rad/s) & \(\zeta\)\\
		24 & 10 & 10 & 10 & 6455 & 1.10\\
		24 & 10 & 22 & 47 & 6348 & 0.51\\
		24 & 10 & 47 & 22 & 634 & 0.24\\
		\hline
	\end{tabular}
\end{table}


\section{結果}

\section{考察}

\section{結論}


さらなる検討の例\cite{huga}も存在するが、この説明では不十分とする場合もある\cite{hoge}。

\bibliographystyle{junsrt}
\bibliography{reference}

\end{document}