\documentclass[../../master.tex]{subfiles}

\graphicspath{{./image/}}

\begin{document}

\chapter{多極子展開}
\section{古典多極子}
\subsection{電気多極子}
\subsubsection*{スカラーポテンシャル}
cgs-Gauss 単位におけるスカラーポテンシャルのポアソン方程式は
\begin{equation}
    \laplacian \phi(\vb*{r}) = -4\pi\rho(\vb*{r})
\end{equation}
である。これの解はルジャンドル多項式と球面調和関数の加法定理を用いて
\begin{align}
    \phi(\vb*{r})
    &= \int d\vb*{r} \frac{\rho(\vb*{r'})}{\abs{\vb*{r}-\vb*{r'}}}\\
    &= \sum_{l=0}^{\infty} \frac{1}{r^{l+1}} \int d\vb*{r'} {r'}^{l} P_l(\cos\theta')\\
    &= \sum_{l=0}^{\infty} \sum_{m=-l}^{l} \frac{1}{r^{l+1}}\int d\vb*{r'} \rho(\vb*{r'}){r'}^l Z_l^m(\theta,\,\varphi) Z_l^{m*} (\theta',\,\varphi')\\
    &= \sum_{l=0}^{\infty} \sum_{m=-l}^{l} \frac{1}{r^{l+1}}Z_l^m(\theta,\,\varphi) \qty[\int d\vb*{r'} O_l^{m} (\theta',\,\varphi')\rho(\vb*{r'})]\\
    &=:\sum_{l=0}^{\infty} \sum_{m=-l}^{l} \frac{1}{r^{l+1}}Z_l^m(\theta,\,\varphi) Q_l^m
\end{align}
というように整理できる。
Notation が増えて申しわけないが、
途中で導入した\(O_l^m(\theta,\,\varphi),\,Q_l^m\)は
\begin{align}
    O_l^m(\vb*{r}) &:= r^l Z_l^{*}(\theta,\,\varphi) = \sqrt{\frac{4\pi}{2l+1}}r^lY_l^{m*}(\theta,\,\varphi)\\
    Q_l^m &:= \int d\vb*{r'} O_l^{m} (\theta',\,\varphi') \rho(\vb*{r'})
\end{align}
というのを導入した。
これは電気多極子モーメントになっている。
実際展開して計算してみよう。
\(l=0,\,m=0\)は
\begin{align}
    Q_0^0
    &= \int d\vb*{r'}  Z_0^0 (\theta',\,\varphi') \rho(\vb*{r'})\\
    &= \int d\vb*{r'} \rho(\vb*{r'})
\end{align}
で全電荷になっている。
電気双極子に相当するものとして
\(l=1\)のものがある。
\(l=1,\,m = 1\)では
\begin{align}
    Q_1^{ 1}
    &= \int d\vb*{r'}  r' Z_1^{1*} (\theta',\,\varphi') \rho(\vb*{r'})\\
    &= \int d\vb*{r'} \qty{- \frac{r'\sin\theta'}{\sqrt{2}}e^{- i\varphi'}} \rho(\vb*{r'})\\
    &= \int d\vb*{r'} \qty{- \frac{x+iy}{\sqrt{2}}} \rho(\vb*{r'}),
\end{align}
\(l=1,\,m=0\)では
\begin{align}
    Q_1^0
    &= \int d\vb*{r'}  r' Z_1^0 (\theta',\,\varphi') \rho(\vb*{r'})\\
    &= \int d\vb*{r'}  r' \cos\theta' \rho(\vb*{r'})\\
    &= \int d\vb*{r'} z \rho(\vb*{r}),
\end{align}
\(l=1,\,m=-1\) では
\begin{align}
    Q_1^{- 1}
    &= \int d\vb*{r'}  r' Z_1^{- 1*} (\theta',\,\varphi') \rho(\vb*{r'})\\
    &= \int d\vb*{r'} \frac{r'\sin\theta'}{\sqrt{2}}e^{- i\varphi'} \rho(\vb*{r'})\\
    &= \int d\vb*{r'} \frac{x+iy}{\sqrt{2}} \rho(\vb*{r'}),
\end{align}
というようになっている。
これは
\(m=0\)では電気双極子モーメントの\(z\)成分、
\(m=1\)では電気双極子モーメントの\(-(x+iy)/\sqrt{2}\)成分、
\(m=-1\)では電気双極子モーメントの\((x-iy)/\sqrt{2}\)成分、
を表している。
ここで変な方向の成分が出てきた。
これは Spherical basis と呼ばれる基底である。
通常の光を直線変更で表したものに対し、円偏光で表したものに相当すると思われる。

% % TODO なんで相当するものになるかの説明
% 書き下すと\(z\)方向に進む光は
% \begin{align}
%     E(z,\,t)
%     &= \vb*{e}_x E_x \cos(kz-\omega t) + \vb*{e}_y E_y \cos(kz-\omega t)\\
%     &= -\frac{\vb*{e}_x+i\vb*{e}_y}{\sqrt{2}}\frac{-E_x+iE_y}{\sqrt{2}} \cos(kz-\omega)
%     + \frac{\vb*{e}_x - i\vb*{e}_y}{\sqrt2}\frac{E_x+iE_y}{\sqrt{2}} \cos(kz-\omega t)
% \end{align}

また、上の計算を見てわかるように\(O_l^{m*}(\vb*{r})\)というのが
多極子モーメントの形を決めるのがわかる。
\(l=2\)の電気四重極子の形を見るため、\(O_l^m\)を整理していこう。
\(l=2,\,m=2\)のとき、
\begin{align}
    O_2^{2}(\vb*{r})
    &= r^2\sqrt{\frac{4\pi}{5}}Y_2^{2*}(\theta,\,\varphi)\\
    &= r^2\sqrt{\frac{3}{8}} \sin^2\theta(\cos\varphi - i\sin\varphi)^2\\
    &= r^2\sqrt{\frac{3}{8}}\sin^2\theta(\cos^2\varphi-\sin^2\varphi - 2i\cos\varphi\sin\varphi)\\
    &=\sqrt{\frac{3}{8}}(x^2-y^2-2ixy)
    =\sqrt{\frac{3}{8}}(x-iy)^2,
\end{align}
\(l=2,\,m=1\)のとき
\begin{align}
    O_2^{1}(\vb*{r})
    &= r^2 \sqrt{\frac{4\pi}{5}}Y_2^{1*}(\theta,\,\varphi)
    = -\sqrt{\frac{3}{2}}r^2\sin\theta\cos\theta (\cos\varphi - i\sin\varphi)\\
    &= \sqrt{\frac{3}{2}}(-xz+iyz) = -\sqrt{\frac{3}{2}}z(x-iy),
\end{align}
\(l=2,\,m=0\)のとき
\begin{align}
    O_2^{0}
    &= r^2 \sqrt{\frac{4\pi}{5}}Y_2^{0*}(\theta,\,\varphi)\\
    &= \frac{r^2}{2}(3\cos^2\theta-1) = \frac{3z^2-r^2}{2},
\end{align}
\(l=2,\,m=-1\)のとき
\begin{align}
    O_2^{-1}(\vb*{r})
    &= r^2 \sqrt{\frac{4\pi}{5}}Y_2^{-1*}(\theta,\,\varphi)\\
    &= \sqrt{\frac{3}{2}}r^2\sin\theta\cos\theta (\cos\varphi + i\sin\varphi)\\
    &= \sqrt{\frac{3}{2}}(xz+iyz) = \sqrt{\frac{3}{2}}z(x+iy),
\end{align}
\(l=2,\,m=-2\)のとき、
\begin{align}
    O_2^{-2}(\vb*{r})
    &= r^2\sqrt{\frac{4\pi}{5}}Y_2^{2*}(\theta,\,\varphi)
    = r^2\sqrt{\frac{3}{8}} \sin^2\theta(\cos\varphi + i\sin\varphi)^2\\
    &= r^2\sqrt{\frac{3}{8}} \sin^2\theta(\cos^2\varphi-\sin^2\varphi + 2i\cos\varphi\sin\varphi)\\
    &=\sqrt{\frac{3}{8}}(x^2-y^2+2ixy) = \sqrt{\frac{3}{8}}(x+iy)^2,
\end{align}
というようになっていく。
くどいようだが、最終的には\(f\)電子の物性を考えたいため\(l=3\)の電気八極子も展開していく。
\(l=3,\,m=3\)のとき
\begin{align}
    O_3^3(\vb*{r})
    &= r^3 \sqrt{\frac{4\pi}{7}} Y_3^{3*}(\theta,\,\varphi)
    = -\frac{\sqrt{5}r^3}{4} \sin^3\theta (\cos\varphi-i\sin\varphi)^3\\
    &= -\frac{\sqrt{5}r^3}{4} \sin^3\theta (\cos^3\varphi - 3\cos\varphi\sin^2\varphi + i(\sin^3\varphi - 3\cos^2\varphi\sin\varphi))\\
    &= -\frac{\sqrt{5}}{4} \Big\{x^3-3xy^2+i(y^3- 3x^2y)\Big\}
    = -\frac{\sqrt{5}}{4} (x-iy)^3,
\end{align}
\(l=3,\,m=2\)のとき
\begin{align}
    O_3^2(\vb*{r})
    &= r^3 \sqrt{\frac{4\pi}{7}} Y_3^{2*}(\theta,\,\varphi)
    = \frac{\sqrt{30}r^3}{8} \sin^2\theta \cos\theta (\cos\varphi-i\sin\varphi)^2\\
    &= \frac{\sqrt{30}r^3}{8} \sin^2\theta \cos\theta (\cos^2\varphi - \sin^2\varphi - 2i\cos\varphi\sin\varphi)\\
    &= \frac{\sqrt{30}}{8} \Big\{zx^2-zy^2 -2ixyz\Big\}
    = \frac{\sqrt{30}}{8} z(x-iy)^2,
\end{align}
\(l=3,\,m=1\)のとき
\begin{align}
    O_3^1(\vb*{r})
    &= r^3 \sqrt{\frac{4\pi}{7}} Y_3^{1*}(\theta,\,\varphi)\\
    &= -r^3 \sqrt{\frac{3}{16}} \sin\theta(5\cos^2\theta-1)(\cos\varphi-i\sin\varphi)\\
    &= -r^3 \sqrt{\frac{3}{16}} \sin\theta(5\cos^2\theta-1)
    \Bigl\{\cos\varphi-i\sin\varphi\Bigr\}\\
    &=-\sqrt{\frac{3}{16}}(5z^2-r^2)(x-iy),
\end{align}
\(l=3,\,m=0\)のとき
\begin{align}
    O_3^(\vb*{r})
    &= r^3 \sqrt{\frac{4\pi}{7}} Y_3^{0}(\theta,\,\varphi)\\
    &= \frac{r^3}{2}  (5\cos^2\theta-3\cos\theta)\\
    &= \frac{z(5z^2-3r^2)}{2},
\end{align}
\(l=3,\,m=-1\)のとき
\begin{align}
    O_3^{-1}(\vb*{r})
    &= r^3 \sqrt{\frac{4\pi}{7}} Y_3^{-1*}(\theta,\,\varphi)\\
    &= r^3 \sqrt{\frac{3}{16}} \sin\theta(5\cos^2-1)(\cos\varphi+i\sin\varphi)\\
    &= r^3 \sqrt{\frac{3}{16}} \sin\theta(5\cos^2-1)
    \Bigl\{\cos\varphi+i\sin\varphi\Bigr\}\\
    &=-\sqrt{\frac{3}{16}}(5z^2-r^2)(x+iy),
\end{align}
\(l=3,\,m=-2\)のとき
\begin{align}
    O_3^{-2}(\vb*{r})
    &= r^3 \sqrt{\frac{4\pi}{7}} Y_3^{-2*}(\theta,\,\varphi)
    = \frac{\sqrt{30}r^3}{8} \sin^2\theta \cos\theta (\cos\varphi+i\sin\varphi)^2\\
    &= \frac{\sqrt{30}r^3}{8} \sin^2\theta \cos\theta (\cos^2\varphi - \sin^2\varphi + 2i\cos\varphi\sin\varphi)\\
    &= \frac{\sqrt{30}}{8} \Big\{zx^2-zy^2 + 2ixyz\Big\}
    = \frac{\sqrt{30}}{8} z(x+iy)^2,
\end{align}
\(l=3,\,m=-3\)のとき
\begin{align}
    O_3^3(\vb*{r})
    &= r^3 \sqrt{\frac{4\pi}{7}} Y_3^{3*}(\theta,\,\varphi)
    = \frac{\sqrt{5}r^3}{4} \sin^3\theta (\cos\varphi+i\sin\varphi)^3\\
    &= \frac{\sqrt{5}r^3}{4} \sin^3\theta (\cos^3\varphi - 3\cos\varphi\sin^2\varphi + i(-\sin^3\varphi - 3\cos^2\varphi\sin\varphi))\\
    &= \frac{\sqrt{5}}{4} \Big\{x^3-3xy^2+i(-y^3-+3x^2y)\Big\}
    = \frac{\sqrt{5}}{4} (x+iy)^3,
\end{align}
というようになる。
これで多極子展開したときのスカラーポテンシャル
\begin{equation}
    \phi(\vb*{r}) = \sum_{l=0}^{\infty} \sum_{m=-l}^{l} Q_l^m\frac{Z_l^m(\theta,\,\varphi)}{r^{l+1}}
\end{equation}
の中身がわかった。

\subsubsection*{電場}
スカラーポテンシャルがわかったら、電場を調べたくなるものである。
なので電場を調べよう。
電場はスカラーポテンシャルの勾配であるので、
多極子展開したスカラーポテンシャルをいれて計算していくと
\begin{align}
    \vb*{E}(\vb*{r}) &= -\grad \phi(\vb*{r})\\
    &= \sum_{l=0}^{\infty} \sum_{m=-l}^{l} Q_l^m \grad\qty(\frac{Z_l^m(\theta,\,\varphi)}{r^{l+1}})\\
    &= \sum_{l=0}^{\infty} \sum_{m=-l}^{l} Q_l^m \qty(-\frac{(l+1)Z_l^m(\theta,\,\varphi)}{r^{l+2}}\vb*{e}_r+\frac{\grad Z_l^m(\theta,\,\varphi)}{r^{l+1}})\\
\end{align}
ここで球面調和関数の微分を行う前に角運動量演算子を思い出そう。
3次元極座標における角運動量演算子は
\begin{align}
    \hat{\vb*{L}} &= \hat{\vb*{r}} \times \hat{\vb*{p}}\\
    &= -i\qty(\vb*{e}_\varphi \pdv{\theta} -\vb*{e}_\theta \frac{1}{\sin\theta}\pdv{\varphi})
\end{align}
であった。
これを次のようにする。
\begin{align}
    \frac{i}{r}\vb*{e}_r \times \hat{\vb*{L}} &= \frac{1}{r}\qty(\vb*{e}_\theta \pdv{\theta} +\vb*{e}_\varphi \frac{1}{\sin\theta}\pdv{\varphi})
\end{align}
これはまさに3次元極座標における勾配の\(\theta,\,\varphi\)成分である。
よって電場は
\begin{equation}
    \vb*{E}(\vb*{r}) = \sum_{l=0}^{\infty} \sum_{m=-l}^{l}\frac{Q_l^m}{r^{l+2}}
    \qty(\frac{(l+1)\vb*{r}Z_l^m(\theta,\,\varphi)+i\vb*{r}\times\qty(\hat{\vb*{L}}Z_l^m(\theta,\,\varphi))}{r})
\end{equation}
となる。後ろの括弧で囲まれた項がベクトル球面調和関数の1つとして知られ、
都合によりこれを\(\sqrt{l+1}\)で割ったもの、
\begin{align}
    \vb*{Z}_{lm}^{(l+1)}(\theta,\,\varphi) &:=-\frac{(l+1)\vb*{r}Z_l^m(\theta,\,\varphi)+i\vb*{r}\times\qty(\hat{\vb*{L}}Z_l^m(\theta,\,\varphi))}{r\sqrt{2l+1}}\\
    \vb*{Y}_{lm}^{(l+1)}(\theta,\,\varphi) &:=-\frac{(l+1)\vb*{r}Y_l^m(\theta,\,\varphi)+i\vb*{r}\times\qty(\hat{\vb*{L}}Y_l^m(\theta,\,\varphi))}{r\sqrt{(l+1)(2l+)}}\\
\end{align}
で定義される。これを使うと電場は
\begin{align}
    \vb*{E}(\vb*{r})
    &= \sum_{l=0}^{\infty} \sum_{m=-l}^{l}\frac{Q_l^m}{r^{l+2}} \vb*{Z}_{lm}^{(l+1)}(\theta,\,\varphi)\\
    &= \sum_{l=0}^{\infty} \sum_{m=-l}^{l}\sqrt{4\pi(l+1)}\frac{Q_l^m}{r^{l+2}} \vb*{Y}_{lm}^{(l+1)}(\theta,\,\varphi)\\
\end{align}
と書ける。


\end{document}