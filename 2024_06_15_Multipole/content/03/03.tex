\documentclass[../../master.tex]{subfiles}

\graphicspath{{./image/}}

\begin{document}

\chapter{多極子展開}
\section{電気多極子}
\subsection{スカラーポテンシャル}
cgs-Gauss 単位におけるスカラーポテンシャルのポアソン方程式は
\begin{equation}
    \laplacian \phi(\vb*{r}) = -4\pi\rho(\vb*{r})
\end{equation}
である。これの解はルジャンドル多項式と球面調和関数の加法定理を用いて
\begin{align}
    \phi(\vb*{r})
    &= \int d\vb*{r'} \frac{\rho(\vb*{r'})}{\abs{\vb*{r}-\vb*{r'}}}
    = \sum_{l=0}^{\infty} \frac{1}{r^{l+1}} \int d\vb*{r'} \rho(\vb*{r'}) {r'}^{l} P_l(\cos\theta')\\
    &= \sum_{l=0}^{\infty} \sum_{m=-l}^{l} \frac{1}{r^{l+1}}\int d\vb*{r'} \rho(\vb*{r'}){r'}^l Z_l^m(\theta,\,\varphi) Z_l^{m*} (\theta',\,\varphi')\\
    &= \sum_{l=0}^{\infty} \sum_{m=-l}^{l} \frac{1}{r^{l+1}}Z_l^m(\theta,\,\varphi) \qty[\int d\vb*{r'} O_l^{m} (\theta',\,\varphi')\rho(\vb*{r'})]\\
    &=:\sum_{l=0}^{\infty} \sum_{m=-l}^{l} \frac{1}{r^{l+1}}Z_l^m(\theta,\,\varphi) Q_l^m
\end{align}
というように整理できる。
Notation が増えて申しわけないが、
途中で導入した\(O_l^m(\theta,\,\varphi),\,Q_l^m\)は
\begin{align}
    O_l^m(\vb*{r}) &:= r^l Z_l^{*}(\theta,\,\varphi) = \sqrt{\frac{4\pi}{2l+1}}r^lY_l^{m*}(\theta,\,\varphi)\\
    Q_l^m &:= \int d\vb*{r'} O_l^{m} (\theta',\,\varphi') \rho(\vb*{r'})
\end{align}
というのを導入した。
これは電気多極子モーメントになっている。
実際展開して計算してみよう。
\(l=0,\,m=0\)は
\begin{align}
    Q_0^0
    = \int d\vb*{r'}  Z_0^0 (\theta',\,\varphi') \rho(\vb*{r'})
    = \int d\vb*{r'} \rho(\vb*{r'})
\end{align}
で全電荷になっている。
電気双極子に相当するものとして
\(l=1\)のものがある。
\(l=1,\,m = 1\)では
\begin{align}
    Q_1^{ 1}
    &= \int d\vb*{r'}  r' Z_1^{1*} (\theta',\,\varphi') \rho(\vb*{r'})\\
    &= \int d\vb*{r'} \qty{- \frac{r'\sin\theta'}{\sqrt{2}}e^{- i\varphi'}} \rho(\vb*{r'})\\
    &= \int d\vb*{r'} \qty{- \frac{x+iy}{\sqrt{2}}} \rho(\vb*{r'}),
\end{align}
\(l=1,\,m=0\)では
\begin{align}
    Q_1^0
    &= \int d\vb*{r'}  r' Z_1^0 (\theta',\,\varphi') \rho(\vb*{r'})\\
    &= \int d\vb*{r'}  r' \cos\theta' \rho(\vb*{r'})\\
    &= \int d\vb*{r'} z \rho(\vb*{r}),
\end{align}
\(l=1,\,m=-1\) では
\begin{align}
    Q_1^{- 1}
    &= \int d\vb*{r'}  r' Z_1^{- 1*} (\theta',\,\varphi') \rho(\vb*{r'})\\
    &= \int d\vb*{r'} \frac{r'\sin\theta'}{\sqrt{2}}e^{- i\varphi'} \rho(\vb*{r'})\\
    &= \int d\vb*{r'} \frac{x+iy}{\sqrt{2}} \rho(\vb*{r'}),
\end{align}
というようになっている。
これは
\(m=0\)では電気双極子モーメントの\(z\)成分、
\(m=1\)では電気双極子モーメントの\(-(x+iy)/\sqrt{2}\)成分、
\(m=-1\)では電気双極子モーメントの\((x-iy)/\sqrt{2}\)成分、
を表している。
ここで変な方向の成分が出てきた。
これは Spherical basis と呼ばれる基底である。
通常の光を直線変更で表したものに対し、円偏光で表したものに相当すると思われる。

% % TODO なんで相当するものになるかの説明
% 書き下すと\(z\)方向に進む光は
% \begin{align}
%     E(z,\,t)
%     &= \vb*{e}_x E_x \cos(kz-\omega t) + \vb*{e}_y E_y \cos(kz-\omega t)\\
%     &= -\frac{\vb*{e}_x+i\vb*{e}_y}{\sqrt{2}}\frac{-E_x+iE_y}{\sqrt{2}} \cos(kz-\omega)
%     + \frac{\vb*{e}_x - i\vb*{e}_y}{\sqrt2}\frac{E_x+iE_y}{\sqrt{2}} \cos(kz-\omega t)
% \end{align}

また、上の計算を見てわかるように\(O_l^{m*}(\vb*{r})\)というのが
多極子の形を決めるのがわかる。
\(l=2\)の電気四重極子の形を見るため、\(O_l^m\)を整理していこう。
\(l=2,\,m=2\)のとき、
\begin{align}
    O_2^{2}(\vb*{r})
    &= r^2\sqrt{\frac{4\pi}{5}}Y_2^{2*}(\theta,\,\varphi)\\
    &= r^2\sqrt{\frac{3}{8}} \sin^2\theta(\cos\varphi - i\sin\varphi)^2\\
    &= r^2\sqrt{\frac{3}{8}}\sin^2\theta(\cos^2\varphi-\sin^2\varphi - 2i\cos\varphi\sin\varphi)\\
    &=\sqrt{\frac{3}{8}}(x^2-y^2-2ixy)
    =\sqrt{\frac{3}{8}}(x-iy)^2,
\end{align}
\(l=2,\,m=1\)のとき
\begin{align}
    O_2^{1}(\vb*{r})
    &= r^2 \sqrt{\frac{4\pi}{5}}Y_2^{1*}(\theta,\,\varphi)
    = -\sqrt{\frac{3}{2}}r^2\sin\theta\cos\theta (\cos\varphi - i\sin\varphi)\\
    &= \sqrt{\frac{3}{2}}(-xz+iyz) = -\sqrt{\frac{3}{2}}z(x-iy),
\end{align}
\(l=2,\,m=0\)のとき
\begin{align}
    O_2^{0}
    &= r^2 \sqrt{\frac{4\pi}{5}}Y_2^{0*}(\theta,\,\varphi)\\
    &= \frac{r^2}{2}(3\cos^2\theta-1) = \frac{3z^2-r^2}{2},
\end{align}
\(l=2,\,m=-1\)のとき
\begin{align}
    O_2^{-1}(\vb*{r})
    &= r^2 \sqrt{\frac{4\pi}{5}}Y_2^{-1*}(\theta,\,\varphi)\\
    &= \sqrt{\frac{3}{2}}r^2\sin\theta\cos\theta (\cos\varphi + i\sin\varphi)\\
    &= \sqrt{\frac{3}{2}}(xz+iyz) = \sqrt{\frac{3}{2}}z(x+iy),
\end{align}
\(l=2,\,m=-2\)のとき、
\begin{align}
    O_2^{-2}(\vb*{r})
    &= r^2\sqrt{\frac{4\pi}{5}}Y_2^{2*}(\theta,\,\varphi)
    = r^2\sqrt{\frac{3}{8}} \sin^2\theta(\cos\varphi + i\sin\varphi)^2\\
    &= r^2\sqrt{\frac{3}{8}} \sin^2\theta(\cos^2\varphi-\sin^2\varphi + 2i\cos\varphi\sin\varphi)\\
    &=\sqrt{\frac{3}{8}}(x^2-y^2+2ixy) = \sqrt{\frac{3}{8}}(x+iy)^2,
\end{align}
というようになっていく。
くどいようだが、最終的には\(f\)電子の物性を考えたいため\(l=3\)の電気八極子も展開していく。
\(l=3,\,m=3\)のとき
\begin{align}
    O_3^3(\vb*{r})
    &= r^3 \sqrt{\frac{4\pi}{7}} Y_3^{3*}(\theta,\,\varphi)
    = -\frac{\sqrt{5}r^3}{4} \sin^3\theta (\cos\varphi-i\sin\varphi)^3\\
    &= -\frac{\sqrt{5}r^3}{4} \sin^3\theta (\cos^3\varphi - 3\cos\varphi\sin^2\varphi + i(\sin^3\varphi - 3\cos^2\varphi\sin\varphi))\\
    &= -\frac{\sqrt{5}}{4} \Big\{x^3-3xy^2+i(y^3- 3x^2y)\Big\}
    = -\frac{\sqrt{5}}{4} (x-iy)^3,
\end{align}
\(l=3,\,m=2\)のとき
\begin{align}
    O_3^2(\vb*{r})
    &= r^3 \sqrt{\frac{4\pi}{7}} Y_3^{2*}(\theta,\,\varphi)
    = \frac{\sqrt{30}r^3}{8} \sin^2\theta \cos\theta (\cos\varphi-i\sin\varphi)^2\\
    &= \frac{\sqrt{30}r^3}{8} \sin^2\theta \cos\theta (\cos^2\varphi - \sin^2\varphi - 2i\cos\varphi\sin\varphi)\\
    &= \frac{\sqrt{30}}{8} \Big\{zx^2-zy^2 -2ixyz\Big\}
    = \frac{\sqrt{30}}{8} z(x-iy)^2,
\end{align}
\(l=3,\,m=1\)のとき
\begin{align}
    O_3^1(\vb*{r})
    &= r^3 \sqrt{\frac{4\pi}{7}} Y_3^{1*}(\theta,\,\varphi)\\
    &= -r^3 \sqrt{\frac{3}{16}} \sin\theta(5\cos^2\theta-1)(\cos\varphi-i\sin\varphi)\\
    &=-\sqrt{\frac{3}{16}}(5z^2-r^2)(x-iy),
\end{align}
\(l=3,\,m=0\)のとき
\begin{align}
    O_3^(\vb*{r})
    &= r^3 \sqrt{\frac{4\pi}{7}} Y_3^{0}(\theta,\,\varphi)\\
    &= \frac{r^3}{2}  (5\cos^2\theta-3\cos\theta)\\
    &= \frac{z(5z^2-3r^2)}{2},
\end{align}
\(l=3,\,m=-1\)のとき
\begin{align}
    O_3^{-1}(\vb*{r})
    &= r^3 \sqrt{\frac{4\pi}{7}} Y_3^{-1*}(\theta,\,\varphi)\\
    &= r^3 \sqrt{\frac{3}{16}} \sin\theta(5\cos^2-1)(\cos\varphi+i\sin\varphi)\\
    &=-\sqrt{\frac{3}{16}}(5z^2-r^2)(x+iy),
\end{align}
\(l=3,\,m=-2\)のとき
\begin{align}
    O_3^{-2}(\vb*{r})
    &= r^3 \sqrt{\frac{4\pi}{7}} Y_3^{-2*}(\theta,\,\varphi)
    = \frac{\sqrt{30}r^3}{8} \sin^2\theta \cos\theta (\cos\varphi+i\sin\varphi)^2\\
    &= \frac{\sqrt{30}r^3}{8} \sin^2\theta \cos\theta (\cos^2\varphi - \sin^2\varphi + 2i\cos\varphi\sin\varphi)\\
    &= \frac{\sqrt{30}}{8} \Big\{zx^2-zy^2 + 2ixyz\Big\}
    = \frac{\sqrt{30}}{8} z(x+iy)^2,
\end{align}
\(l=3,\,m=-3\)のとき
\begin{align}
    O_3^3(\vb*{r})
    &= r^3 \sqrt{\frac{4\pi}{7}} Y_3^{3*}(\theta,\,\varphi)
    = \frac{\sqrt{5}r^3}{4} \sin^3\theta (\cos\varphi+i\sin\varphi)^3\\
    &= \frac{\sqrt{5}r^3}{4} \sin^3\theta (\cos^3\varphi - 3\cos\varphi\sin^2\varphi + i(-\sin^3\varphi - 3\cos^2\varphi\sin\varphi))\\
    &= \frac{\sqrt{5}}{4} \Big\{x^3-3xy^2+i(-y^3-+3x^2y)\Big\}
    = \frac{\sqrt{5}}{4} (x+iy)^3,
\end{align}
というようになる。
これで多極子展開したときのスカラーポテンシャル
\begin{equation}
    \phi(\vb*{r}) = \sum_{l=0}^{\infty} \sum_{m=-l}^{l} Q_l^m\frac{Z_l^m(\theta,\,\varphi)}{r^{l+1}}
\end{equation}
の中身がわかった。

\subsection{電場}
スカラーポテンシャルがわかったら、電場を調べたくなるものである。
なので電場を調べよう。
電場はスカラーポテンシャルの勾配であるので、
多極子展開したスカラーポテンシャルをいれて計算していくと
\begin{align}
    \vb*{E}(\vb*{r}) &= -\grad \phi(\vb*{r})
    = -\sum_{l=0}^{\infty} \sum_{m=-l}^{l} Q_l^m \grad\qty(\frac{Z_l^m(\theta,\,\varphi)}{r^{l+1}})\\
    &= -\sum_{l=0}^{\infty} \sum_{m=-l}^{l} Q_l^m \qty(-\frac{(l+1)Z_l^m(\theta,\,\varphi)}{r^{l+2}}\vb*{e}_r+\frac{\grad Z_l^m(\theta,\,\varphi)}{r^{l+1}})\\
\end{align}
ここで球面調和関数の微分を行う前に角運動量演算子を思い出そう。
3次元極座標における角運動量演算子は
\begin{align}
    \hat{\vb*{L}} &= \hat{\vb*{r}} \times \hat{\vb*{p}}
    = -i\qty(\vb*{e}_\varphi \pdv{\theta} -\vb*{e}_\theta \frac{1}{\sin\theta}\pdv{\varphi})
\end{align}
であった。
これを次のようにする。
\begin{align}
    -\frac{i}{r^2}\vb*{r} \times \hat{\vb*{L}} &= \frac{1}{r}\qty(\vb*{e}_\theta \pdv{\theta} +\vb*{e}_\varphi \frac{1}{\sin\theta}\pdv{\varphi})
\end{align}
これはまさに3次元極座標における勾配の\(\theta,\,\varphi\)成分である。
よって電場は
\begin{equation}
    \vb*{E}(\vb*{r}) = -\sum_{l=0}^{\infty} \sum_{m=-l}^{l}\frac{Q_l^m}{r^{l+2}}
    \qty(-\frac{(l+1)\vb*{r}Z_l^m(\theta,\,\varphi)+i\vb*{r}\times\qty(\hat{\vb*{L}}Z_l^m(\theta,\,\varphi))}{r})
\end{equation}
となる。後ろの括弧で囲まれた項がベクトル球面調和関数の1つとして知られ、
都合によりこれを\(\sqrt{l+1}\)で割ったもの、
\begin{align}
    \vb*{Z}_{lm}^{(l+1)}(\theta,\,\varphi) &:=-\frac{(l+1)\vb*{r}Z_l^m(\theta,\,\varphi)+i\vb*{r}\times\qty(\hat{\vb*{L}}Z_l^m(\theta,\,\varphi))}{r\sqrt{l+1}}\\
    \vb*{Y}_{lm}^{(l+1)}(\theta,\,\varphi) &:=-\frac{(l+1)\vb*{r}Y_l^m(\theta,\,\varphi)+i\vb*{r}\times\qty(\hat{\vb*{L}}Y_l^m(\theta,\,\varphi))}{r\sqrt{(l+1)(2l+1)}}\\
\end{align}
で定義される。
\footnote{楠瀬先生は\(Z_{lm}^{(l)}\)は使っておらず、勝手に定義したものなのであまり一般的ではないかもしれない。}
これを使うと電場は
\begin{align}
    \vb*{E}(\vb*{r})
    = -\sum_{l=0}^{\infty} \sum_{m=-l}^{l}\sqrt{l+1}\frac{Q_l^m}{r^{l+2}} \vb*{Z}_{lm}^{(l+1)}(\theta,\,\varphi)
    = -\sum_{l=0}^{\infty} \sum_{m=-l}^{l}\sqrt{4\pi(l+1)}\frac{Q_l^m}{r^{l+2}} \vb*{Y}_{lm}^{(l+1)}(\theta,\,\varphi)\\
\end{align}
と書ける。

\subsection{電気多極子のパリティ}
電気多極子の定義は
\begin{equation}
    Q_l^m = \int d\vb*{r}\, r^l Z_l^m(\theta,\,\varphi)\rho(\vb*{r})
\end{equation}
であった。空間反転では\(\theta\rightarrow\pi-\theta,\quad\varphi\rightarrow\varphi\pm 2\pi\)としてやればよいので、
空間反転に関するパリティは球面調和関数による\((-1)^l\)である。
時間反転に関しては、電荷は時間反転に関して偶ハリティであるため全体としても偶パリティである。

\section{磁気多極子}
\subsection{ベクトルポテンシャル}
磁場に関するガウスの法則
\begin{equation}
    \div \vb*{B}(\vb*{r}) = 0
\end{equation}
より
\begin{equation}
    \vb*{B}(\vb*{r}) = \curl \vb*{A}(\vb*{r})
\end{equation}
となるベクトル関数\(\vb*{A}(\vb*{r})\)がある。
これをベクトルポテンシャルという。
これを cgs-Gauss のアンペールの式に入れると
\begin{align}
    \curl (\curl \vb*{A}(\vb*{r})) &= \frac{4\pi}{c} \vb*{j}(\vb*{r})\\
    \grad(\div\vb*{A}(\vb*{r})) - \laplacian \vb*{A}(\vb*{r}) &= \frac{4\pi}{c}\vb*{j}(\vb*{r})\\
    \laplacian \vb*{A}(\vb*{r}) &= -\frac{4\pi}{c}\vb*{j}(\vb*{r})
\end{align}
最後はクーロンゲージ条件
\begin{equation}
    \div \vb*{A}(\vb*{r})=0
\end{equation}
を用いた。
これを多重極展開を用いた解は
\begin{align}
    \vb*{A}(\vb*{r})
    &= \frac{1}{c}\int d\vb*{r'} \frac{\rho(\vb*{r'})}{\abs{\vb*{r}-\vb*{r'}}}\\
    &= \frac{1}{c}\sum_{l=0}^{\infty} \frac{1}{r^{l+1}}\int d\vb*{r'} \vb*{j}(\vb*{r'}){r'}^l P_l(\cos\theta')\\
    &= \frac{1}{c}\sum_{l=0}^{\infty} \frac{1}{r^{l+1}} Z_l^m(\theta,\,\varphi)\int d\vb*{r'} \vb*{j}(\vb*{r'}){r'}^l Z_l^{m*}(\theta',\,\varphi')\\
    &= \sum_{l=0}^{\infty} \sum_{m=-l}^{l} \frac{1}{cr^{l+1}}Z_l^m(\theta,\,\varphi) \int d\vb*{r'} \vb*{j}(\vb*{r'}) O_l^{m}(\vb*{r'})
\end{align}
となる。
このまま積分の中身を\(J_{lm}\)と定義するという案もあるが、
どうやらベクトルポテンシャルのゲージ不変性で不都合が出るらしい\cite{Stefan2016}。
総和の中は以下のように変形できる。

以下この節は工事中

\clearpage

目標:
\begin{align}
    \frac{1}{cr^{l+1}} Z_l^m(\theta,\,\varphi) &\int d\vb*{r'} \vb*{j}(\vb*{r'}) O_l^{m}(\vb*{r'})\notag\\
    &\quad=\sqrt{\frac{l+1}{l}}\quad\underset{=M_m^l}{\underline{\frac{1}{c(l+1)}\int d\vb*{r}[\vb*{r}\times \vb*{j}(\vb*{r})]\cdot \grad O_l^m(\vb*{r})}}
    \quad\frac{1}{ir^{l+1}}\quad\underset{=\vb*{Z}_{lm}^{(l)}(\vb*{r})}{\underline{\frac{\hat{\vb*{L}}Z_l^m(\theta,\,\varphi)}{\sqrt{l(l+1)}}}}\notag\\
    &\quad-\sqrt{l+1}\underset{=T_m^l}{\underline{\frac{1}{c(l+1)}\int d\vb*{r}[\vb*{r}\cdot \vb*{j}(\vb*{r})] O_l^m(\vb*{r})}}
    \quad\underset{=\grad(Z_l^m(\theta,\,\varphi)/r^{l+1})}{\underline{\frac{1}{r^{l+2}}\underset{=\vb*{Z}_{lm}^{(l+1)}}{\underline{\qty[-\frac{1}{r}\frac{(l+1)\vb*{r}Z_l^m(\theta,\,\varphi)+i\vb*{r}\times [\hat{\vb*{L}}Z_l^m(\theta,\,\varphi)]}{\sqrt{l+1}}]}}}}
\end{align}
% 電流密度は磁気モーメントをもちいて
% \begin{equation}
%     \vb*{j}(\vb*{r}) = c \curl \vb*{M}(\vb*{r})
% \end{equation}
% と書ける。
% これより
% \begin{equation}
%     \frac{1}{c}\int d\vb*{r}\, \vb*{j}(\vb*{r}) r^l Z_l^{m*}(\theta,\,\varphi)
%     = \int d\vb*{r}\, \qty(\curl\vb*{M}(\vb*{r})) O_l^m (\theta,\,\varphi).
% \end{equation}
変形していく。
\begin{align}
    \int d\vb*{r'} \vb*{j}(\vb*{r'}) O_l^{m}(\vb*{r'})
    &= \vb*{e}_i \int d\vb*{r'} j_i O_l^{m}\\
    &= \vb*{e}_i \int d\vb*{r'} \nabla'\cdot(x_i' \vb*{j}) O_l^m\\
    &= -\int d\vb*{r'} \vb*{r'} (\vb*{j}\cdot \nabla' O_l^m)\\
    &= \int d\vb*{r'} (\vb*{r'}\times\vb*{j})\times \nabla' O_l^m - \int d\vb*{r'} (\vb*{r'}\cdot\vb*{j})\nabla'O_l^m
\end{align}
ここで、
以前やった
\begin{equation}
    \grad Z_l^m = \frac{-i}{r^2}\vb*{r}\times \hat{\vb*{L}}Z_l^m
\end{equation}
に注意すると\(\grad O_l^m\)は
\begin{align}
    \grad O_l^m
    &= \grad(r^l Z_l^m)\\
    &= lr^{l-1}Z_l^m \vb*{e}_r +r^l\qty(-\frac{i}{r^2}\vb*{r}\times \hat{\vb*{L}}Z_l^m)\\
    &= r^{l-2}\qty(l\vb*{r}Z_l^m-i\vb*{r}\times\hat{\vb*{L}}Z_l^m)\\
    &=\frac{l\vb*{r}O_l^m-i\vb*{r}\times \hat{\vb*{L}}O_l^m}{r^2}
\end{align}
と書けるので、第一項の中身は
\begin{align}
    (\vb*{r}\times\vb*{j})\times\frac{l\vb*{r}O_l^m-i\vb*{r}\times \hat{\vb*{L}}O_l^m}{r^2}
    &=-\frac{(\vb*{r}\times\vb*{j})\times(-i\vb*{r}\times\hat{\vb*{L}}O_l^m)}{r^2}\\
    &= \frac{(\vb*{r}\times\vb*{j})\cdot\hat{\vb*{L}}Z_l^m}{r^2}\vb*{r}\\
    &=-(\vb*{r}\times \vb*{j})\cdot(l\vb*{r}Z_l^m)
\end{align}


\clearpage
\section{電荷磁荷対応}
これまでに3種類の多極子が現れたが、
時間空間反転対称性の偶奇性から物性を整理することを考えると4種類の多極子があることが望ましい。
では4種類目はどこから現れるかというと、
電荷\(\rho\)と電流\(\vb*{j}\)を磁荷\(\rho_m\)と磁化電流密度\(\vb*{j}_m\)に置き換えたものを考えることで得られる。

\subsection{電気多極子\(\leftrightarrow\)磁気多極子}
まず初めに電気多極子の電荷密度を置き換えたものを見る。
電荷と分極密度\(\vb*{P}\)の間には
\begin{equation}
    \rho(\vb*{r}) = -\div\vb*{P}(\vb*{r})
\end{equation}
という関係があるので、これとのアナロジーで
磁荷と磁化密度\(\vb*{M}\)の間には
\begin{equation}
    \rho_m(\vb*{r}) = -\div\vb*{M}(\vb*{r})
\end{equation}
という関係が成り立っているとする。
このとき、電気多極子の電荷を磁荷に置き換えたものを\(\tilde{Q}_l^m\)とすると
\begin{align}
    \tilde{Q}_l^m
    = \int d\vb*{r}\, O_l^m(\vb*{r})\rho_m(\vb*{r})
    = -\int d\vb*{r}\, O_l^m(\vb*{r})\div\vb*{M}(\vb*{r})
    = \int d\vb*{r}\, \vb*{M}(\vb*{r})\cdot \nabla O_l^m(\vb*{r})
\end{align}
ここである恒等式を示す。
まず
\begin{align}
    \frac{1}{c}\int d\vb*{r}\, (\vb*{r}O_l^m(\vb*{r}))\cdot (\curl (\curl\vb*{M}(\vb*{r})))
    &=\int d\vb*{r}\, (\vb*{r}O_l^m(\vb*{r}))\cdot(\curl \vb*{j}(\vb*{r}))
    =\int d\vb*{r}\, \varepsilon_{ijk} x_i O_l^m(\vb*{r}) \partial_j j_k(\vb*{r})\\
    &= \int d\vb*{r}\,\varepsilon_{ijk} \Biggl[\partial_j \Bigl\{ x_i O_l^m(\vb*{r}) j_k(\vb*{r})\Bigr\}
    -  \varepsilon_{ijk}\partial_j\Bigl\{x_i O_l^m(\vb*{r})\Bigr\} j_k(\vb*{r})\Biggr]\\
    &=\int d\vb*{r}\varepsilon_{ikj} x_i j_k(\vb*{r}) (\partial_j O_l^m(\vb*{r}))
    =\int d\vb*{r}\, (\vb*{r}\times\vb*{j}(\vb*{r}))\cdot \grad O_l^m(\vb*{r})
\end{align}
というように変形できる。
また
\begin{align}
    \int d\vb*{r}\, (\vb*{r}O_l^m(\vb*{r}))\cdot&(\curl (\curl\vb*{M}(\vb*{r}))) \notag\\
    &=\int d\vb*{r}\, \varepsilon_{ijk}\varepsilon_{kpq} x_i O_l^m \partial_j \partial_p M_q\\
    &=\int d\vb*{r}\, \varepsilon_{ijk}\varepsilon_{kpq}
    \qty{\partial_j(x_i O_l^m \partial_p M_q)-\partial_j(x_i O_l^m)\partial_p M_q}\\
    &=\int d\vb*{r}\, \varepsilon_{ijk}\varepsilon_{kpq}
    \qty{-\partial_p(\partial_j(x_i O_l^m)\partial_p M_q)+ \partial_p\partial_j(x_iO_l^m)M_q}\\
    &=\int d\vb*{r}\, \varepsilon_{ijk}\varepsilon_{kpq}
    \partial_j(\delta_{ip}O_l^m + x_i\partial_pO_l^m)M_q\\
    &=\int d\vb*{r}\, \varepsilon_{pjk}\varepsilon_{kpq} (\partial_j O_l^m)M_q
    +\int d\vb*{r}\, (\delta_{ip}\delta_{jq}-\delta_{iq}\delta_{jp})\partial_j(x_i\partial_p O_l^m)M_q\\
    &=2\int d\vb*{r}\, \vb*{M}\cdot\grad O_l^m(\vb*{r})
    + \int d\vb*{r}\, \vb*{M} \cdot \Bigl\{\div(\vb*{r}\cdot \grad O_l^m(\vb*{r}))\Bigr\}
    -\int d\vb*{r}\, \partial_p(x_q\partial_p O_l^m) M_q
\end{align}
というように変形できる。
第二項の積分の中身は
\begin{align}
    \div(\vb*{r}\cdot \grad O_l^m(\vb*{r}))
    = \div(\vb*{r}\pdv{r} r^l Z_l^m(\theta,\,\varphi))
    = \div(\vb*{e}_r \, lr^{l} Z_l^m(\theta,\,\varphi))
    = l\grad O_l^m(\theta,\,\varphi)
\end{align}
第三項の積分の中身は
\begin{align}
    \partial_p(x_q\partial_p O_l^m) M_q
    = (\partial_p x_q) \partial_p O_m^l M_q + x_q M_q \partial_p \partial_p O_l^m
    = \vb*{M} \cdot \grad O_l^m(\vb*{r}) + \vb*{r}\cdot \laplacian O_m^l(\vb*{r})
\end{align}
となる。\(O_l^m(\vb*{r})\)はラプラス方程式の解であるのでこれの第二項は 0 となる。
以上より
\begin{align}
    \int d\vb*{r}\, (\vb*{r}O_l^m(\vb*{r}))\cdot&(\curl (\curl\vb*{M}(\vb*{r})))
    = (l+1)\int d\vb*{r}\, \vb*{M}(\vb*{r})\cdot\grad O_l^m(\vb*{r})
\end{align}
が言える。
示したかった恒等式
\begin{equation}
    c(l+1)\int d\vb*{r}\, \vb*{M}(\vb*{r})\cdot\grad O_l^m(\vb*{r})
    =\int d\vb*{r}\, (\vb*{r}\times\vb*{j}(\vb*{r}))\cdot \grad O_l^m(\vb*{r})
\end{equation}
が導けた。
これをもともと考えてた式に入れると
\begin{align}
    \tilde{Q}_l^m &= \int d\vb*{r}\, \vb*{M}(\vb*{r}) \cdot \grad O_l^m(\vb*{r}) \\
    &=\frac{1}{c(l+1)}\int d\vb*{r}\,(\vb*{r}\times \vb*{j}(\vb*{r}))\cdot \grad O_l^m(\vb*{r}) = M_l^m
\end{align}
というように磁気多極子を得られた。

では電流を磁流に置き換える。
電流は\(\vb*{j}=c\curl \vb*{M}\)であるので、
磁流を\(\vb*{j}_m=c\curl \vb*{P}\)であると考える。
\(M_l^m\)の電流を磁流に置き換えたものを\(\tilde{M}_l^m\)とすると
先ほどの変形を使って
\begin{align}
    \tilde{M}_l^m
    &= \frac{1}{c(l+1)}\int d\vb*{r}\, (\vb*{r}\times \vb*{j}_m(\vb*{r}))\cdot\grad O_l^m\\
    &= \frac{1}{l+1}\int d\vb*{r}\, (\vb*{r}\times (\curl \vb*{P}(\vb*{r}))) \cdot\grad O_l^m\\
    &= \int d\vb*{r}\, \,\vb*{P}(\vb*{r}) \cdot \grad O_l^m(\vb*{r})
    = \int d\vb*{r}\, \,\rho(\vb*{r}) O_l^m(\vb*{r})
\end{align}
というように電気多極子\(Q_l^m\)が得られる。
最後は部分積分をして、\(\rho = -\div \vb*{P}\)を使った。

\subsection{電気トロイダル多極子}
では、相方がまだ出ていない磁気トロイダル多極子に対しても電荷・磁荷変換を行う。
そうしたものを\(\tilde{T}_l^m\)として、
\(\vb*{P} =\curl \vb*{G}\)という量も導入すると、
\begin{align}
    \tilde{T}_l^m
    &= \frac{1}{c(l+1)} \int d\vb*{r}\, (\vb*{r}\cdot \vb*{j}_m(\vb*{r}))O_l^m(\vb*{r})
    = \frac{1}{l+1} \int d\vb*{r}\, (\vb*{r}\cdot (\curl \vb*{P}(\vb*{r})))O_l^m(\vb*{r})\\
    &= \frac{1}{l+1} \int d\vb*{r}\, \varepsilon_{ijk} \partial_j(r_i P_k O_l^m(\vb*{r}))
    - \frac{1}{l+1} \int d\vb*{r}\, \varepsilon_{ijk} \partial_j(r_i) P_k O_l^m(\vb*{r})
    - \frac{1}{l+1} \int d\vb*{r}\, \varepsilon_{ijk} r_i P_k \partial_jO_l^m(\vb*{r})\\
    &= \frac{1}{l+1} \int d\vb*{r}\, (\vb*{r}\times \vb*{P}(\vb*{r})) \cdot \grad O_l^m(\vb*{r})
    = \frac{1}{l+1} \int d\vb*{r}\, (\vb*{r}\times (\curl \vb*{G}(\vb*{r}))) \cdot \grad O_l^m(\vb*{r})\\
    &= \int d\vb*{r}\, \,\vb*{G}(\vb*{r}) \cdot \grad O_l^m(\vb*{r})
    \equiv G_l^m
\end{align}
というようになにか得られた。
\(G_l^m\)を電気トロイダル多極子、
\(\vb*{G}\)を電気トロイダル分極と呼ぶ。

電荷・磁荷変換を施したマクスウェル方程式は
\begin{align}
    \div \vb*{E} &= 0\\
    \div \vb*{B} &= 4\pi \rho_m\\
    \curl \vb*{E} &= \frac{4\pi}{c}\vb*{j}_m\\
    \curl \vb*{B} &= 0
\end{align}
というようになる。
この式を見ると
\(\phi(\vb*{r}),\,\vb*{A}(\vb*{r})\)を電荷・磁荷変換を施したポテンシャル
\(\tilde{\phi}(\vb*{r}),\,\vb*{\tilde{A}}(\vb*{r})\)は
\begin{align}
    \vb*{B} = -\grad{\tilde{\phi}(\vb*{r})} \qquad
    \vb*{E} = \curl \vb*{\tilde{A}}(\vb*{r})
\end{align}
となる。
これを解くと
\begin{align}
    \tilde{\phi}(\vb*{r})
    &= \sum_{l=0}^{\infty} \sum_{m=-l}^{l} M_l^m \frac{Z_l^m(\theta,\,\varphi)}{r^{l+1}}\\
    \vb*{\tilde{A}}(\vb*{r})
    &= \sum_{l=0}^{\infty} \sum_{m=-l}^{l}\qty[
        \sqrt{\frac{4\pi(l+1)}{(2l+1)l}}Q_l^m \frac{\vb*{Y}_{lm}^{(l)}}{ir^{l+1}}
        -\sqrt{4\pi(l+1)} G_l^m \frac{\vb*{Y}_{lm}^{(l+1)}}{r^{l+2}}
    ]
\end{align}
となることが予想される。

\section{まとめ}
いろいろな量が出てきた。それらの関係をまとめておこう。
電荷・磁荷・電流・磁流、分極・磁化・磁気トロイダル分極・電気トロイダル分極
の関係は次のようになっている。
\begin{align}
    \rho(\vb*{r}) = -\div \vb*{P}(\vb*{r}), \quad & \quad \rho_m(\vb*{r}) = -\div \vb*{M}(\vb*{r})\\
    \vb*{j}(\vb*{r}) = c\curl \vb*{M}(\vb*{r}), \quad & \quad \vb*{j}_m(\vb*{r}) = c \curl \vb*{P}(\vb*{r})\\
    \vb*{M}(\vb*{r}) = \curl \vb*{T}(\vb*{r}), \quad & \quad \vb*{P}(\vb*{r}) = \curl \vb*{G}(\vb*{r})\\
    \vb*{T} = \frac{1}{l+1}\, \vb*{r}\times\vb*{M}(\vb*{r}), \quad & \quad \vb*{G}(\vb*{r}) = \frac{1}{l+1}\, \vb*{r}\times\vb*{P}(\vb*{r})
\end{align}
こうして得られた多極子をまとめると、
\begin{align}
    &Q_l^m
    = \int d\vb*{r}\, \rho(\vb*{r})O_l^m(\vb*{r})
    = \int d\vb*{r}\, \vb*{P}(\vb*{r}) \cdot \grad O_l^m(\vb*{r})
    = \frac{1}{c(l+1)}\int d\vb*{r}\, (\vb*{r}\times \vb*{j}_m(\vb*{r})) \cdot \grad O_l^m(\vb*{r})\\
    &M_l^m
    = \int d\vb*{r}\, \rho_m(\vb*{r})O_l^m(\vb*{r})
    = \int d\vb*{r}\, \vb*{M}(\vb*{r}) \cdot \grad O_l^m(\vb*{r})
    = \frac{1}{c(l+1)}\int d\vb*{r}\, (\vb*{r}\times \vb*{j}(\vb*{r})) \cdot \grad O_l^m(\vb*{r})\\
    &T_l^m
    = \int d\vb*{r}\, \vb*{T}(\vb*{r})\cdot \grad O_l^m(\vb*{r})
    = \frac{1}{l+1}\int d\vb*{r}\, (\vb*{r} \times \vb*{M}(\vb*{r}))\cdot \grad O_l^m(\vb*{r})
    = \frac{1}{c(l+1)}\int d\vb*{r}\, (\vb*{r}\cdot \vb*{j}(\vb*{r})) O_l^m(\vb*{r})\\
    &G_l^m
    = \int d\vb*{r}\, \vb*{G}(\vb*{r})\cdot \grad O_l^m(\vb*{r})
    = \frac{1}{l+1}\int d\vb*{r}\, (\vb*{r} \times \vb*{P}(\vb*{r}))\cdot \grad O_l^m(\vb*{r})
    = \frac{1}{c(l+1)}\int d\vb*{r}\, (\vb*{r}\cdot \vb*{j}_m(\vb*{r})) O_l^m(\vb*{r})
\end{align}
また、空間反転に関するパリティは
\begin{align}
    \mathcal{P} \rho = + \rho, \qquad
    \mathcal{P} \rho_m = - \rho_m, \qquad&
    \mathcal{P} \vb*{j} = - \vb*{j}, \qquad
    \mathcal{P} \vb*{j}_m = + \vb*{j}_m\\
    \mathcal{P}Q_l^m = (-1)^{l}Q_l^m,\qquad
    \mathcal{P}M_l^m = (-1)^{l+1}M_l^m,\qquad&
    \mathcal{P}T_l^m = (-1)^{l}T_l^m,\qquad
    \mathcal{P}G_l^m = (-1)^{l+1}G_l^m
\end{align}
時間反転に関するパリティは
\begin{align}
    \mathcal{T} \rho = + \rho, \qquad
    \mathcal{T} \rho_m = - \rho_m, \qquad&
    \mathcal{T} \vb*{j} = - \vb*{j}, \qquad
    \mathcal{T} \vb*{j}_m = + \vb*{j}_m\\
    \mathcal{T}Q_l^m = +Q_l^m,\qquad
    \mathcal{T}M_l^m = -M_l^m,\qquad&
    \mathcal{T}T_l^m = -T_l^m,\qquad
    \mathcal{T}G_l^m = +G_l^m
\end{align}
となる。

\section{分極密度による表示(おまけ)}
あまりモチベーションはわからないが、トロイダル分極を密度表示するというのも考えられる。
\begin{equation}
    \rho_t(\vb*{r}) = -\div \vb*{T}(\vb*{r}) \qquad
    \rho_g(\vb*{r}) = -\div \vb*{G}(\vb*{r}) \qquad
\end{equation}
これを用いて磁気トロイダル多極子と電気トロイダル多極子を表示すると
\begin{align}
    &T_l^m
    = \int d\vb*{r}\, \vb*{T}(\vb*{r})\cdot \grad O_l^m(\vb*{r})
    = \int d\vb*{r}\, \rho_t(\vb*{r}) O_l^m(\vb*{r})\\
    &G_l^m
    = \int d\vb*{r}\, \vb*{G}(\vb*{r})\cdot \grad O_l^m(\vb*{r})
    = \int d\vb*{r}\, \rho_g(\vb*{r}) O_l^m(\vb*{r})
\end{align}
となる。これらトロイダル分極密度のパリティは
\begin{align}
    \mathcal{P} \rho_t = + \rho_t, \qquad&
    \mathcal{P} \rho_g = - \rho_g \\
    \mathcal{T} \rho_t = - \rho_t, \qquad&
    \mathcal{T} \rho_g = + \rho_g
\end{align}
となる。
また、密度があるなら流れもあるだろうということで流れを考えると
\begin{equation}
    \vb*{j}_t = c \curl \vb*{G} (\vb*{r}) = c \vb*{P} \qquad
    \vb*{j}_g = c \curl \vb*{T} (\vb*{r}) = c \vb*{M} \qquad
\end{equation}
というようになる。

\section{コメント}
もともとポアソン方程式だけ見ると引数や演算子に時間は入ってこないため、
時間反転というのは考えることはできない。
どこから時間反転の話が入ってくるかというと、ソースである電流から来ているように見える。
またある意味、単磁荷を導入したのも時間反転を導入するためである。
なんというか、ここら辺は物理的直観で導出したなんとも不思議な性質である。

相対論的電磁気学という面から多極子を見るというのもありなのだろうか?
一種の極限でこのような性質が見れるというのもあるし、
モノポールを入れた電磁気の一種として見るのもおもしいのかもしれない。
また、角運動量演算子のような位置と微分が絡まった演算子がさらにバリエーションが増えるのも考えられる。

\end{document}