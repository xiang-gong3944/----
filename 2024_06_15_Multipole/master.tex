\documentclass[11pt,dvipdfmx,a4paper]{jsreport}

\usepackage[margin=15mm]{geometry} %余白の削除

\usepackage{subfiles}

\usepackage{amsmath,amssymb}
\usepackage{bm}
\usepackage[dvipdfmx]{graphicx}
\usepackage{physics} % http://mirrors.ibiblio.org/CTAN/macros/latex/contrib/physics/physics.pdf
\usepackage{siunitx} %SI単位を楽に出力
\usepackage{mathtools} %環境の追加
\usepackage{pdfpages}
\usepackage[subrefformat=parens]{subcaption} %(a)図のようにすることができるやつ
\usepackage{here}
\usepackage{mathrsfs}
\usepackage{url}
% \usepackage{circuitikz} %電気回路をtex中で書く
% \usepackage{caption} %番号なしキャプションを書く
% \usepackage{cancel} %式中に斜線を入れる
% \usepackage{tensor} %テンソルの添え字を書く
% \usepackage{tikz} %図を書く
% \usepackage{ascmac} %四角い枠の中に文章を書く
% \usepackage{float} %figureで[hbp]オプションを使う
% \usepackage{hyperref}  \usepackage{pxjahyper} %ハイパーリンクをつかう
% \usepackage{tablefootnote} %表中に注釈をいれる
% \usepackage[thicklines]{cancel} %数式中の取り消し線
% \usepackage[version=4]{mhchem} %化学式の入力
% \usepackage{wrapfig} %文章の回り込み

% \graphicspath{{./image/}}

% \numberwithin{equation}{section} %式番号を(セクション.式番号)にする

\setcounter{tocdepth}{2}

\title{多極子と物性物理}
\author{東京理科大学理学部第一部応用物理学科 西原翔}
\date{\today}

\begin{document}

% \twocolumn

% \maketitle

% 2024年度東京理科大学理学部第一部応用物理学科遠山研プレ配属にて、
% 雑誌固体物理の多極子の連載を読むゼミを行った。
% その際、関連することや記述の足りない部分をこの文章にまとめることにした。

% 筆者がこの文章を書き始めたときに持っている知識としては、
% 多重極放射は砂川理論電磁気学\cite{Sunakawa}で一度眺めたことはある。
% そこでは電気四重極子、磁気双極子放射までしかやっておらずこの内容をやるうえでは
% あと少しものたりないような感じであった。
% ジャクソン\cite{Jackson}にも書いてあるそうだが、分厚い鈍器みたいな本であるためまだ追えていない。
% 量子力学については
% 前期量子論から水素原子までやったことはあるものの、
% 特殊関数が苦手すぎて、球面調和関数はちゃんとやったことはなかった。
% 球面調和関数がわからず、角運動量についての記述がわからないところもあったため、
% 角運動量にの合成といった話も分からない。
% また、量子化学で扱うような束縛状態の他電子系については一応授業でやっているのだが、
% ちゃんと身にはなっていない。そのため 3d 電子系の話や結晶点群による既約分解等はさっぱりわからない。

% といったように前提知識が足りていないような人が書いた文章である。
% そのため、少なくとも自分にはわかりやすく書いたつもりであるが、
% 説明が誤っていたりする箇所が多々あると思われる。
% そのような点や計算ミス等を見つけた際には
% xiang248a@gmail.com にメールを出していただけると嬉しい。

% % この本はまず初めに球面調和関数についての章から始める。
% % 多極子を扱う上で球面調和関数を覚えておかないとどうしようもない箇所があるので、
% % ルジャンドル多項式の母関数がをポアソン方程式の解になっているところから始めた。
% % そこから、級数表示、漸化式、微分方程式、ロドリゲスの公式、直交性というように進めていく。
% % その後、改めて3次元極座標ラプラシアンでポアソン方程式を解き、ルジャンドル陪多項式と球面調和関数を導出していく。
% % 級数解からだと、わかりにくくなるため量子力学の角運動量の持つ代数的構造がこの微分方程式に使えることに触れ、
% % 昇降演算子による3次元極座標におけるポアソン方程式の一般解を導出した。
% % 球面調和関数の直交性と加法定理を示してこの章は終わる。
% % あまりなじみのない Notation や 角運動量代数と球面調和関数の関係についても触れられているので
% % のぞいていってほしい。

% % 次の章では角運動量の合成とそれに伴う性質を触れていく。

% % 第3章ではいよいよ多極子を扱っていく。
% % まず初めに古典的な多極子放射として、
% % 電気多極子からはじめ、磁気多極子、磁気トロイダル多極子の放射を計算していく。
% % その後、電荷磁荷対応に触れることで、
% % 磁荷と磁流が電気トロイダル多極子を作ることを見ていく。

% ゼミにて指導してくださった、遠山先生、議論に付き合ってくれた同期の上原君に感謝する。

% \clearpage

% \tableofcontents
% \clearpage

% コメントアウトにより master に入れるか入れないかを決める。
\clearpage
\subfile{./content/01/01.tex}
% \clearpage
% \subfile{./content/02/02.tex}
\clearpage
\subfile{./content/03/03.tex}
\clearpage
\subfile{./content/04/04.tex}
% 一通り書いたら、この構成に書き直したいね
%
% part 1 準備
% 1章 : ルジャンドル多項式
% 2章 : 角運動量の一般論
% 3章 : 軌道角運動量と球面調和関数
% 4章 : スピン角運動量と空間回転
% 5章 : 角運動量の合成
% 6章 : テンソル演算子
% 7章 : 量子化学と点群
% part 2 多極子
% 8章 : 古典多極子放射
% 9章



\bibliographystyle{junsrt}
\bibliography{reference}

\end{document}